\documentclass{article}

\title{CS466: Design and Analysis of Algorithms \\ Assignment 2 \\ Spring 2016 \\ Instructor: Mark Petrick}
\author{MingHao Lu \\ 20465562}
\setlength\parindent{0pt}
\usepackage{amsmath}
\usepackage[margin=1in]{geometry}

\begin{document}
    \pagenumbering{gobble}
    \maketitle
    \newpage
    \pagenumbering{arabic}

\section*{Question 1}
    Suppose the claim is not true, then the amortized cost must be of  $o(\log(k))$. 
    Consider the following sequence of operations. Given a list of n number, perform n insertions
    and then n deletemin. Since the average over this sequence(amortized cost of any operation) is $o(\log(n))$ 
    and this sequence has 2*n operations, the total cost must be $2n*o(\log(n))$ or $o(n\log(n))$.
    This sequence of insertion and deletion gives us the n numbers in increasing order, effectively sorting them.
    Since the keys can only be accessed by pairwise comparison, this is a comparison
    sorting algorithm of $o(\log(n))$. However, we know comparison based sorting algorithms
    has a lower bound of $\Omega(\log(n))$. Contradiction. Therefore the claim must be true.

\section*{Question 2}
\subsection*{i)}
    Linking( adding one tree as a leftmost child of another) is obviously constant time. Insertion
    is just linking the heap with a new heap of one node. Merging is a linking. Decrease key is removing
    a tree(same as removing a child from the parent node, which is constant time) and a linking.
    Since these operations only involve one or two constant time operations, they are O(1). \\

    The only steps to take during deletemin is to delete the root node and consolidate all
    of its children into a single tree. Suppose the root has r children, then after removing
    the root you are left with r seperate trees. It'd take a constant time operation to
    decrease the number of trees left by one, so total cost is $\Theta(r)$.
    Since $ r \le n$, and if the tree has only two layers then $r=n-1$, we conclude deletemin
    cost $\Theta(n)$ worst case.

\subsection*{ii)}
    cost(i) is the cost for severing the root ($O(1)$) and joining the children of the root 
    together. Let k be the number of children the root has. Each joining operation removes 
    one tree and takes $O(1)$. Since there are k trees and we apply join until only one tree 
    is left, this takes $O(k-1)$. Together with the sever, this is $O(k)$. Since $k \le n$, 
    cost(i) is $O(n)$.\\

    Let $\Phi_i$ be the number of nodes in the tree, $\Phi_0$ is zero and $\Phi_n$ is non-negative.
    Then we have $\Phi_n > \Phi_0$. Change in potential from deletemin is -1.
    $charge(i) = cost(i) + \Phi_i - \Phi_{i-1} = O(n) -1 = O(n)$

\end{document}
